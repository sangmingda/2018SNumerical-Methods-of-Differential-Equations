\documentclass{article}%book,report,letter

\usepackage{ctex}
\usepackage{fontspec}
%\usepackage{color}
%\usepackage{graphicx} %use graph format
%\usepackage{subfigure}
%\usepackage{epstopdf} %eps图片
\usepackage{amsmath}  %字体加粗
%\usepackage{math}
\usepackage{amsthm}
\usepackage{amssymb} %因为所以符号
%\usepackage{caption}
%\captionsetup[table]{labelsep=space}
\usepackage{float}%图片位置

%自定义命令
\newcommand*{\myTestTimes}{1\xspace}
%\typein[\myTestTimes]{这是第几次测试?}
\newcommand*{\myName}{桑明达\xspace}
\newcommand*{\myNumber}{15300180062\xspace}
\newcommand*{\myHomeworkNumber}{第十四周作业\xspace}
\newcommand*{\myArticleName}{微分方程数值解法\xspace}

\newcommand*{\myseries}[2][n]{\ensuremath{#2_1,#2_2,\dots,#2_{#1}}}


%制作页眉页脚
\usepackage{fancyhdr}
\pagestyle{fancy}
\lhead{\myHomeworkNumber}
\chead{\myArticleName}
\rhead{\myName \myNumber}
\lfoot{}
\cfoot{\thepage}
\rfoot{}
\renewcommand{\headrulewidth}{0.4pt}
\renewcommand{\footrulewidth}{0.4pt}

%标题
\title{\heiti \myArticleName \\ [2ex] \begin{large} \myHomeworkNumber \end{large}}
\author{\kaishu \myName \myNumber}
\date{\today}

% 正文区
\begin{document}
\maketitle
%\newpage
\section{Douglas-Rachford格式稳定性分析}

\begin{proof}
	对于Douglas-Rachford格式,使用Von Neumann传播因子法分析,有
	\begin{align*}
		\frac{u^{n+1}-u^{n}}{\tau}=&\Delta_hu^{n+1}-\tau\delta^2_x\delta^2_y(u^{n+1}-u^{n}) \\
		\text{记}\\
		u^n_{i,j}=&g(n)e^{i(\omega _xx_i+\omega_yy_j)} \\
		\Delta_hu^n_{i,j}=&g(n)e^{i(\omega _xx_i+\omega_yy_j)}\left (\frac{e^{i\omega_xh_x}+e^{-i\omega_xh_x}}{h^2_x}+\frac{e^{i\omega_yh_y}+e^{-i\omega_yh_y}}{h^2_y}  \right ) \\
    =&-4g(n)e^{i(\omega _xx_i+\omega_yy_j)}\left (\frac{\sin ^2\frac{\omega_xh_x}{2}}{h^2_x}+\frac{\sin ^2\frac{\omega_yh_y}{2}}{h^2_y}  \right ) \\
		\text{类似,有}\\
		\delta_xu^n_{i,j}=&-4g(n)e^{i(\omega _xx_i+\omega_yy_j)}\frac{\sin ^2\frac{\omega_xh_x}{2}}{h^2_x}\\
    \delta_yu^n_{i,j}=&-4g(n)e^{i(\omega _yy_i+\omega_yy_j)}\frac{\sin ^2\frac{\omega_yh_y}{2}}{h^2_y}\\
    \delta_x\delta_yu^n_{i,j}=&16g(n)e^{i(\omega _yy_i+\omega_yy_j)}\frac{\sin ^2\frac{\omega_xh_x}{2}}{h^2_x}\frac{\sin ^2\frac{\omega_yh_y}{2}}{h^2_y}\\
\end{align*}
\begin{align*}
		\text{代入格式左边,有}\\
		\text{left}=&\frac{g(n+1)e^{i(\omega _xx_i+\omega_yy_j)}-g(n)e^{i(\omega _xx_i+\omega_yy_j)}}{\tau}\\
		\text{代入格式右边,有}\\
    \text{right}=&-4g(n+1)e^{i(\omega _xx_i+\omega_yy_j)}\left (\frac{\sin ^2\frac{\omega_xh_x}{2}}{h^2_x}+\frac{\sin ^2\frac{\omega_yh_y}{2}}{h^2_y}  \right )\\
    &-16\tau (g(n+1)-g(n))e^{i(\omega _yy_i+\omega_yy_j)}\frac{\sin ^2\frac{\omega_xh_x}{2}}{h^2_x}\frac{\sin ^2\frac{\omega_yh_y}{2}}{h^2_y} \\
		\text{移项化简,有}\\
    g(n+1)=&g(n)\frac{1+16\tau^2\frac{\sin ^2\frac{\omega_xh_x}{2}}{h^2_x}\frac{\sin ^2\frac{\omega_yh_y}{2}}{h^2_y}}{1+4\tau \frac{\sin ^2\frac{\omega_xh_x}{2}}{h^2_x}+4\tau  \frac{\sin ^2\frac{\omega_yh_y}{2}}{h^2_y}+16\tau^2\frac{\sin ^2\frac{\omega_xh_x}{2}}{h^2_x}\frac{\sin ^2\frac{\omega_yh_y}{2}}{h^2_y}}\\
		\text{所以,最终得到}\\
		\left |G  \right |=&\left |\frac{1+16\tau^2\frac{\sin ^2\frac{\omega_xh_x}{2}}{h^2_x}\frac{\sin ^2\frac{\omega_yh_y}{2}}{h^2_y}}{1+4\tau \frac{\sin ^2\frac{\omega_xh_x}{2}}{h^2_x}+4\tau  \frac{\sin ^2\frac{\omega_yh_y}{2}}{h^2_y}+16\tau^2\frac{\sin ^2\frac{\omega_xh_x}{2}}{h^2_x}\frac{\sin ^2\frac{\omega_yh_y}{2}}{h^2_y}}  \right | \leq 1\\
\end{align*}
    所以Douglas-Rachford格式无条件稳定。
\end{proof}

\section{P222 2 Leap-frog格式截断误差以及数值稳定性}

\begin{proof}
截断误差为$$ R^n_i=\frac{\tau^2}{6}\frac{\partial^3u}{\partial t^3}(t_n,x_i)+O(\tau^4)+c\frac{h^2}{6}\frac{\partial^3u}{\partial x^3}+O(h^4) $$

对于Leap-frog格式,使用Von Neumann传播因子法分析,有$$ G_\pm =i\frac{c\tau}{h}\sin \omega h \pm \sqrt{1-(\frac{c\tau}{h}\sin \omega h)^2} $$

    所以若$ 1-(\frac{c\tau}{h}\sin \omega h)^2 \geq 0 $,则Leap-frog格式稳定;否则不一定。
\end{proof}

\section{P222 3 Lax-Wendroff格式精度以及数值稳定性}

\begin{proof}
截断误差为$$ R^n_i = \frac{\tau^2}{6}\frac{\partial ^3u}{\partial t^3}(t_n,x_i)+c\frac{h^2}{6}\frac{\partial ^3u}{\partial x^3}(t_n,x_i)+O(\tau^4)+O(h^3)+O(\tau h^3) $$
所以,数值格式是关于时间空间二阶精度的。

对于Lax-Wendroff格式,使用Von Neumann传播因子法分析,有$$ \left |G  \right |^2=1-4r^2\left (1-r^2  \right )\sin^4\frac{\omega h}{2} $$
所以当$ r \leq 0 $时,格式稳定。
\end{proof}

\section{P222 4 Beam-Warming格式精度以及稳定性}

\begin{proof}
截断误差为$$ R^n_i=\frac{\tau^2}{6}\frac{\partial^3u}{\partial t^3}(t_n,x_i)+O(\tau^3)-\frac{ch^2}{3}\frac{\partial^2u}{\partial x^2}+O(h^3)+\frac{c^2\tau h}{2}\frac{\partial^3u}{\partial x^3}+O(\tau h^2) $$

对于Beam-Warming格式,使用Von Neumann传播因子法分析,有$$ \left |G  \right |^2=1+4r(r-1)^2(r-2)\sin^4\frac{\omega h}{2} $$

    所以若$ r \leq 2 $时,$ G \leq 1$,格式稳定。
\end{proof}
\end{document}
